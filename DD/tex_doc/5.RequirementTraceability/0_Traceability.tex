\chapter{Requirements Traceability}
In this part of the \emph{Design Document} are reported the requirements, taken from the \emph{RASD} Document (version 1.2), in order to create a connection with the features, the modules and the components described through the entire document.

\renewcommand{\arraystretch}{1.5}

\begin{table}[H]
    \centering
    \begin{tabular}{p{4cm} | p{7cm}}
        \large{\textbf{Component}} & \large{\textbf{Satisfied Requirements}} \\
        \hline
        
        Sign Up System              & R1, R16                                   \\
        
        Client                      & R14, R15                                  \\
        
        Data Access Manager         & R13, R16, R17                                  \\
        
        Data Layer                  & R9, R11, R16, R17                         \\
        
        Cost Evaluator              & R17, R18                             \\
        
        Business WorkFlow           & R4, R6, R7, R8, R11                       \\
        
        Scheduler                   & R3, R4, R5, R6, R7, R8, R10, R12, R18     \\
        
        Reminder                    & R2, R3                                    \\
    \end{tabular}
    \caption{Main features connected with the satisfied functional requirements}
\end{table}

As can be noticed from the table, there are eight main functionalities of the \emph{Travlendar+} system that are involved in the satisfaction of the requirement. 
First of all, there is the \emph{Sign Up} functionality, that is used to satisfy the requirement R1 (the user should specify a residence at registration time), in this way the system can know where the user starts his day and where he ends it one. This feature also satisfies the requirement R16 (the system should permit a new user signing up through a proper page).

The \emph{Client} component permit the satisfiability of the requirement R14 (The application should permit three different calendar visualization: by month, by week or by day) and also the requirement R15 (The application should permit the selection of a single task, in order to change preferences concerning that task).

The \emph{Data Access Manager} module is used to load and store, in a Database, the main useful business entities, so it satisfies the requirement R16, because the user data should be stored in the Database at signing up time; the requirement R13 (the system should permit the purchase of a either a public transport ticket or a shared-based vehicle service) and the R17 (The system, in order to choose the best travel option, has to consider two main parameters: cost of the travel and the distance between the two locations); these ones are satisfied because the \emph{Data Access Manager} layer is responsible of the communication with the external services, such as the map service or the transport service.

Even if the \emph{Data Layer} is pretty similar to the \emph{Data Access Manager}, it is a layer connected directly to the DBMS service. So it satisfies the requirements R16 and R17, because this layer is useful to store both the user's data and some others data concerning the various travels; in addition this layer is useful to satisfy the requirements R9 (Each FixedTask can be repeated a finite or an infinite number of time) and R11 (The software should permit to modify both the general user's preferences and the tasks' preferences).

The \emph{Cost Evaluator} module is used to have a good value that describes the "goodness" of a schedule, by computing some mathematical functions to have a numerical value used to evaluate it. Moreover this module is used to load the costs of the various public services (such as the public transport service or the shared-based vehicle). So this component satisfies the requirement R17, because it computes effectively the function used to evaluate a single schedule, and the requirement R18 (the system should suggest the best travel ticket, if the public transport service is a user's preference), because this component also compute how many times the user has to take the public transport means, and it knows how much a ticket cost.

The \emph{Business WorkFlow} component is the main responsible of the correct management of the work flow, in order to compute properly the user's calendar. So the requirements it satisfies are: R4 (If there are two or more overlaps, then the application should ask to the user how it will handle this situation), R6 (The system should do a reschedule, if the user modifies, adds or deletes one or more tasks), R7 (If the user adds, deletes o modifies a task during the current day, the application may not be able to schedule a feasible calendar), R8 (If the user adds a task that isn't reachable in a feasible time, the system should send a warning to the user) and R11. All these functional requirements, as can be notice from the \emph{RASD} Document, are all tied to the correct management of the application's (or user's) events.

The \emph{Scheduler} component is used to schedule the user's tasks in a feasible way, and it is the main feature of the \emph{Travlendar+} application. Indeed, it satisfies the most important functional requirements: R3 (If there isn't a shared-based vehicle, the system should compute a feasible alternative), R4 (If there are two or more overlaps, then the application should ask to the user how it will handle this situation), R5 (The software should permit to the user to specify when he wants to come back at home), R6 (The system should do a reschedule, if the user modifies, adds or deletes one or more tasks), R7 (If the user adds, deletes o modifies a task during the current day, the application may not be able to schedule a feasible calendar), R8 (If the user adds a task that isn't reachable in a feasible time, the system should send a warning to the user), R10 (The application should taking into account if the user has taken his private vehicle during the day, in order to permit the user to retake his vehicle), R12 (The application should allow the user to insert tasks which are time-variable during the week) and R18 (the system should suggest the best travel ticket, if the public transport service is a user's preference). In this case, all the requirements are related to the computation of a feasible calendar for the user.

Finally, the \emph{Reminder} module is used to check, if a travel consider to take a shared-based vehicle, the feasibility of that schedules, in order to avoid the situation that, when a user needs to start his trip, there are no vehicles available. So this component satisfies the requirement R2 (The  application  must  send  a  reminder  to  the  user  30  minutes  before  he  should begin the trip.  Moreover, the application has to verify if a Sharing-Based vehicle is effectively available, if actually it is a user preference) and R3 (If there aren't available shared-based vehicle, the system should compute a different travel solution that is feasible). Both of them are related to the reschedule of a trip if there are some issues concerning the shared-based services.