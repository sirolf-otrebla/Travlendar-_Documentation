\section{Document Structure}
The \emph{RASD} Document is divided into 4 main parts:
\begin{itemize}
    \item Introduction
    \item Overall Description
    \item Specific Requirements
    \item Formal Analysis
\end{itemize}
The main purpose of this document is to describe in depth the characteristic of the \emph{Travlendar+} application.

In the first chapter the system is described at an high level of information, explaining the main purpose of the application and its scope. Moreover there are also a part that helps the reader in understanding the main definition, acronyms and abbreviation used in the entire document.

In the second chapter, the document goes more in detail about the application, talking about the main goals the application needs to satisfy, and how the system interacts with the main actors, such as the final user, the public transport service, the car-sharing or the bike-sharing service. Finally in this part there is a section devoted to explain the main domain assumption the \emph{Travlendar+} environment requires to reach the goals.

In the third chapter is reported some mock ups about our application, divided in mobile application and web application, indeed our system can be used either on a mobile device or on a classical web browser. After that there is an explanation about the requirements of the \emph{Travlendar+}, both functional and non-functional, and also there are some \emph{UML Use Case} diagrams and \emph{UML Sequence} diagrams in order to describe better how the application will work.

In the chapter four are attached the \emph{Alloy} models used to verify the correctness of the \emph{Travlendar+} system. In this way we have modeled the main characteristics of our project, and they are all verified with this prover.

Finally there is the chapter five, in which is reported the effort spent by each member of the team.

