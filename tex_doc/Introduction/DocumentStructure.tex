\section{Document Structure}
The \emph{RASD} Document is divided into 4 main parts:
\begin{itemize}
    \item Introduction
    \item Overall Description
    \item Specific Requirements
    \item Formal Analysis
\end{itemize}
The main purpose of this document is to describe in depth the characteristic of the \emph{Travlendar+} application.

In the first chapter is described the System at an high level of information, explaining the main purpose of the application and its scope. Moreover there is also a part that helps the reader in understanding the main definitions, acronyms and abbreviations used in the entire document.

In the second chapter, the document goes more in detail about the application, talking about the main goals the application needs to satisfy, and how the System interacts with the main actors, such as the final user, the public transport service, the car-sharing or the bike-sharing service. Finally, in this part there is a section devoted to explain the main domain assumptions the \emph{Travlendar+} environment requires to reach the goals.

In the third chapter are reported some mock ups about our application, divided in mobile application and web application, as indeed our System can be used either on a mobile device or on a classical web browser. After that there is an explanation about the functional and non-functional requirements of \emph{Travlendar+} and also there are some \emph{UML Use Case} diagrams and \emph{UML Sequence} diagrams that aim to describe better how the application will work.

In the chapter four are attached the \emph{Alloy} models used to verify the correctness of the \emph{Travlendar+} System. The code is reported and properly documented and, in addiction, there are some runs of the more relevant assertions and predicates. 

Finally there is the chapter five, in which is reported the effort spent by each member of the team.

