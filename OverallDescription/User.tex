\section{User Characteristics}
The main actors who interact with the \emph{Travlendar+} software will be presented under this section. Note that the contribute of each actor can vary consistently, depending mainly on implementation and design details. For example, it can be possible to obtain certain data from one actor instead of another, or there can be actors who provide only redundant informations: that can be true assuming that some kind of data can be obtained from multiple sources, and it is easy to understand that at design and implementation level to interact with a smaller number of actors can be preferred.
\begin{itemize}

	\item \emph{Public Transport Services}: The software will have to interact with Public transport services in order to obtain traffic information, news about strikes, bus stops, metro stations and public transportation scheduling.
    
    \item \emph{Map and Weather Services}: The software will have to interact with cloud services that provide weather information. moreover, the software will have to interact with map services such as \emph{Google Maps} in order to schedule tasks and suggest the best itinerary for each travel.
    
    \item \emph{Visitor and Registered User}: The software will have to interact consistently with either visitors ( that are, as described before, unregistered users) and registered users. This means that the software will have to recognize whether a user is registered and logged, and, if not, it will have to provide a sign up and sign in interface. In particular, \emph{Travlendar+} services will be available only for Registered Users, \textcolor{blue}{whether visitors can only visit explanation pages about the service.}
    
    \item \emph{Bike and Car Sharing Services}: The software will have to interact consistently with Bike sharing and Car sharing services in order to provide consistent traveling alternatives to users who can and want to use such services. in particular, the software will have to refer to services \textcolor{blue}{which are present in the deployment location}. 
    
    \item \emph{Local Administrations}: The software will have to interact with local administrations (e.g. municipalities, regional administrations) in order to have access to useful data that can be used in order to offer the best possible service. For example, the software could access to historical traffic data \textcolor{blue}{in order to suggest whether the user should use his personal vehicle or the public transportation service}. 
    
    \item \emph{Mail and External \textcolor{blue}{Calendar} Services}: The software will have to interact with external calendar service in order to import and \textcolor{red}{possibly syncronize the user's task schedule, so that it can be accessed using external applications (for example, using \emph{Google calendar} ).}
    
    \item \emph{System Administrator}: The system administrator is the Actor who takes care of keeping the \emph{Travlendar+} Service active and available. It can be an organization or a single person, and has to guarantee the service's uptime and functionalities as later specified on this document.
\end{itemize}